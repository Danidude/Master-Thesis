\chapter{Conclusion and further work}
\label{ch:conclusion}

\section{Conclusion}

The goal of the project was to both simulate a crisis on board a ship and find safe passage of the ship for the passengers. Our focus was on an Ant Colony Optimization pathfinding algorithm which performance was measured against Dijkstra's pathfinding algorithm.  The algorithms were tested in a dynamic crisis environment where passengers could panic, fire could spread and high occupant density could hamper movements. We worked on a solution where the ships were modeled as graphs with nodes representing rooms and edges representing transitions between rooms. At the start of the simulation fire would break out and it would spread at an increased rate as time passed. Basic human behavior was implemented by giving each passenger an increased chance to panic when in close proximity to danger or other panicking passengers. Additionally family members would stick together and the system would make no attempts to send them in different directions. 

The simulations showed that in small graphs Dijkstra's algorithm would performed better. The time it takes for ACO to calculate a safe passage is tied to the amount of ants the algorithm runs, and a certain amount of ants is necessary to ensure a good result. Thus Dijkstra's algorithm is both faster and saves more lives in a simulation using a small graph. In larger graphs ACO performs faster than Dijkstra's algorithm as the increasing number of edges and nodes impacts Dijkstra's algorithm more than ACO.

As the project is over we can state that the it met the requirements. The algorithms were tested under the conditions that were planned and produced good results. The system is not ready for deployment but it does show that there are benefits to such a system if it were completed. ACO performed well under certain conditions and could be used as the pathfinding algorithm in a future system.

\section{Further Work}

As stated in the introduction, the goal of our work was not to create a fully functional system, one that could be deployed in a real crisis. To do so one would need to expand upon our solution. First the system would need to be able to identify all phones in the crisis area. This is necessary to both be able to identify which phone to send directions to and to track the movements of all passengers. The movements of the passengers would be tracked using the GPS in the phone. Second the system needs to be able to send directions to the phones. One can not expect every passenger to install an app, thus the system would need a way to take over the phone and show the image of the escape route. 

Third the system would need a way to alert all passengers that evacuation routes will be sent to their phones. For instance by displaying messages on TV screens and announcing it over the ships audio systems. In case of electric failure the TV screens and audio system would need to have backup batteries.  Fourth, not all passengers carry smart phones thus the system should send the directions to one phone and have the owner guide the surrounding passengers. In the case where there are passengers that are not carrying a smart phone the system would need an alternate way of tracking passengers, possibly by using on board cameras.

Fifth the system needs would need a database of all ship models. This could be done by creating a parser that transforms ship blueprints into graphs. Any company that wished to have this evacuation system installed would have to submit blueprints of their ships. Sixth the system would need to calculate the routes as fast as possible and as safe as possible. This could be solved by a combination of using massive computer power, optimizing the algorithm and spending a finite amount of time calculating each route.