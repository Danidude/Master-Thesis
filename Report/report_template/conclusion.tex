\chapter{Conclusion and further work}
\label{ch:conclusion}
% 5 pages

\section{Summary of Results}

\section{Conclusion}

\section{Further Work}

As stated in the introduction, the goal of our work was not to create a fully functional system, one that could be deployed in a real crisis. To do so one would need to expand upon our solution. First the system would need to be able to identify all phones in the crisis area. This is necessary to both be able to identify which phone to send directions to and to track the movements of all passengers. The movements of the passengers would be tracked using the GPS in the phone. Second the system needs to be able to send directions to the phones. One can not expect every passenger to install an app, thus the system would need a way to take over the phone and show the image of the escape route. 

Third the system would need a way to alert all passengers that evacuation routes will be sent to their phones. For instance by displaying messages on TV screens and announcing it over the ships audio systems. In case of electric failure the TV screens and audio system would need to have backup batteries.  Fourth, not all passengers carry smart phones thus the system should send the directions to one phone and have the owner guide the surrounding passengers. In the case where there are passengers that are not carrying a smart phone the system would need an alternate way of tracking passengers, possibly by using on board cameras.

Fifth the system needs would need a database of all ship models. This could be done by creating a parser that transforms ship blueprints into graphs. Any company that wished to have this evacuation system installed would have to submit blueprints of their ships. Sixth the system would need to calculate the routes as fast as possible and as safe as possible. This could be solved by a combination of using massive computer power, optimizing the algorithm and spending a finite amount of time calculating each route.

