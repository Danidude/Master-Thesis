\chapter{Background}
\label{ch:background}

The purpose of this chapter is to describe relevant prior work. It will explain the basics of the algorithms
used as well as recent research that expands upon them. Dijkstra's algorithm has been widely used to solve
graph search problems, and as such is a suitable algorithm to compare to the Ant Colony Optimization algorithms.
Human behavior in a crisis is hard to predict because of the complex nature of humans and the lack of data from 
crisis situations. Nevertheless, given a large amount of people, some statistical behavior can be assumed.

\section{Human Behavor Models}
There exists several behavioral models that predict movement. However the most successfully ones predict 
animal movement as animals are rather simple compared to humans. The environment for animals tends to 
be simpler, for instance ocean compared to urban, and animals are simpler creatures and will always rely only 
on instinct, whether in an emergency or not. There are several different computational tools used to simulate  
human movement during emergencies, for instance fluid and particle systems, matrix-based systems and  
emergent systems. Nevertheless they all either have limitations or they are somewhat inaccurate \cite{Pan:2007} \cite{6062828}. 
 
It has been found that fluid systems and cellular automata do not adequately describe human movement \cite{Pelechano2008377} as humans  
panic and make choices, both good and bad, and they all have individual quirks. For instance, in a room with  
multiple exits a crowd tend to form a herd mentality and the majority will gather around a single exit, which  
will slow down their speed, and other exits will see little use. However had this been simulated with a fluid system 
and cellular automata it would likely predict that all exits would be used equally for maximum effect. 
 
In a matrix-based system the room(s) are divided into cells. The cells can represent a table, it can be full of  
people or it can be empty. People can move from cell to cell at a rate described in the model and by the  
rules of the model. It is important that these cells and rules are properly adjusted to fit each model as to  
make it as realistic and precise as possible. 
 
In emergent systems simple parts can interact to simulate complex systems. The existing emergent systems  
are often criticized for being too simple \cite{simple}. The system are often given only a few parameters (for example the  
average speed of the humans and the location of the exits) and then attempt to model the situation by  
assuming all humans move towards the most logical exit. 
 
In general the computational tools rely too much on assumptions and too little on the sociological and psychological  
factors \cite{simple}. When a human is put into a crisis-situation that person can either rely on instinct, experience or making  
choices based on the alternatives. Relying on instinct typically means that the person is panicking and make immediate  
decisions that may or may not be good ones. People have been trampled to death by crowds that desperately is trying  
to get away from danger. 
 
A person will often follow the path it knows the best. If a worker walks up the same flight of stairs to his office  
every day for 5 years it is expected that he will also escape down the same flight of stairs during a crisis. This  
is why there are fire drills, so that during a crisis everyone have walked the quickest paths out. 
 
In a crisis that is time-sensitive, for instance a fire, it is very difficulty for a person to weigh his options and  
choose carefully. Some people will not stand around to observe which path might be the best or whether the  
door is hot before opening it. And standing around and choosing carefully might be even worse as  
over time there might be fewer safe paths as the fire spread. Therefore most people will make quick choices 
and move as fast as they can if danger is imminent. 
 
An individual will likely behave different when he is alone, as opposed to when he is in a group. For example  
a family will likely stick together \cite{Yang2005411} and follow the leader, which is more often than not the father. Thus putting  
his or hers experience and instincts aside and becoming a part of the group. Emotions also have a tendency  
to spread quickly in a group, when some people panic it is likely that more will panic shortly thereafter. If  
there is widespread panic within the group the chance of people pushing down and trample others increase  
significantly. 
 
Xiaoshan Pan et al.\cite{Pan:2007} created a framework to simulate human behavior. This framework implemented some 
basic human sociological and psychological factors. Though in this model the humans were not given a path 
to safety, they were merely left to their own devices. We will create a model where we give the people the 
safest path to the exit. We aim to have the people act as realistic as possible and be placed in a dynamic 
and complex environment.

In Ocean Engineering volume 53, there is an article named "Cell-based evacuation simulation considering human behaviour in a passenger ship"\cite{celleva}. 
The ship was modelled by placing cells representing one space enough to fill one passenger, so each room consists of several cells. 
Each cell had 3 different states it could be in: Occupied, Empty or Object. When considering human behaviour, the humans needed some set of rules, so 3 rules was created: separation, cohesion, and alignment. 
Separation or collision avoidance is that passengers tend to avoid to clash into each other. Cohesion is that the passengers wants to be as close to the center of a group of passengers, as this is more comforting. The last rule, alignment is that the passengers tend to head in the same direction and speed as the other passengers when evacuating.
 
 
\section{ACO}
The Ant Colony Organization(ACO) algorithm is a probabilistic algorithm that have been used in routing, scheduling, 
subset and machine learning. It was created by Marco Dorigo in 1992 to find an optimal path in a graph, mimicking the
behavior of ant which are searching for food \cite{aco}. In nature, when searching for food, ants walk randomly until
they locate a food source. When an ant stumble upon food it returns to the ant colony, taking the shortest path possible.
While returning to its colony it leaves a trail of pheromones which will attract other ants. These other ants will, by walking
the same path as the initial ant, strengthen the pheromone trail and further increase the possibility that more ants will
take this path to the food source.

While the pheromone trail increases the probability that an ant chooses to follow the trail, it does not ensure it.
If several ants reaches the same food source by walking different paths the shortest path will be made the more attractive
one over time. This is a result of pheromone deposits. The shortest path will have the highest pheromone density as more
pheromones can be deposited on the shorter route, the resulting higher pheromone density will increase the possibility that 
other ants will follow the shorter path. 

Over time pheromones evaporates, this prevents the system from ending up with local optimal solutions. 
If the pheromones did not evaporate the path followed by a certain amount of initial ants
would quickly become the only path any ant followed as the pheromone density would increase to the point that no other paths
could ever become more attractive, even if they were the more optimal paths. 

Additionally, pheromone evaporation makes ACO able to adapt to changes. A path can be the optimal path for a while, until a change
occurs that makes the path unusable. If the pheromones did not evaporate the ants would continue to follow this path indefinitely,
never reaching the food source again. However, when the ants are not able to reach the food source they will not deposit any
pheromones and as the pheromones deposited by previous ants evaporates the ants will will be less attracted to the old path
and start exploring new ones.
 

\begin{figure}[h]
\centering
\begin{math}
P^s_{ab} = {(T^\alpha_{ab})(\Lambda^\beta_{ab}) \over \sum (T^\alpha_{ab})(\Lambda^\beta_{ab})}
\end{math}
\caption{The mathematical formula for edge selection}
\label{fig:edge}
\end{figure}

For the algorithm itself there are two parts, edge selection, as seen in figure \ref{fig:edge}, and pheromone update, as seen in figure \ref{fig:update}. 
Edge selection, simply stated, is the process of choosing which way to go. Every ant $s$ has a probability $P^s_{ab}$ of moving from location $a$ to location $b$. The probability
is calculated based on the attractiveness $\Lambda_{ab}$  of moving from $a$ to $b$ and the pheromone trail level $T_{ab}$. The trail level,
as discussed above, signifies how rewarding that path has been in the past. The attractiveness of a move indicates the cost of the move and 
will in the beginning of the simmulation have a large impact on the choices made by the ants. However over time the pheromone trail level will take over as the
dominating factor.

\begin{figure}[h]
\centering
\begin{math}
T_{ab} \leftarrow (1 - \rho)T_{ab} + \sum \limits_{s} \Delta T^s_{ab}
\end{math}
\caption{The mathematical formula for pheromone update}
\label{fig:update}
\end{figure}

When the ants have returned, the pheromone trail is updated. The pheromone trail level $T_{ab}$ is reduced by a pheromone evaporation
coefficient $\rho$. A higher coefficient results in higher evaporation and fast adaptation and visa versa. Nevertheless this does not mean
that the factor should always be high as ACO is probabilistic and fast adaptation could result in the optimal path being found and then lost
again before other ants had time to travel it. Finally, $\sum \limits_{s} \Delta T^s_{ab}$ is the amount of pheromones ant number $s$ will deposit on
the trail. The amount deposited is typically decided by taking a constant $I$ divided by the length of ant $s$ path $J_s$

\begin{figure}[h]
\centering
\begin{math}
\Delta \tau^{s}_{ab} =
\begin{cases}
I/J_s & \mbox{if ant }s\mbox{ uses edge }ab\mbox{ in its path} \\
0 & \mbox{otherwise}
\end{cases}
\end{math}
\caption{The mathematical formula used to calculate the amount of pheromones that should be deposited}
\label{fig:update}
\end{figure}

In the article Ant Colony Organization for Best Path Planning\cite{acobpp:2004}, they used ACO to find the optimal path in a network, and update pheromones based on how long the path is and how heavily the traffic is on that route. By multiplying the amount of traffic in the node with the amount of pheromones, ACO may adapt to the traffic inside the network.

In the research paper Ant Colony Optimization for Planning Safe Escape Routes \cite{acofpser} M. Goodwin et al. describes a general approach to using ACO in crisis situations. In the paper they write that: "Computer and mathematical models have shown to be valuable for escape planning with large complex building with many people but is mainly assuming a static representation of hazards". Our simulations work in a similar way to theirs, by receiving information from the crisis area and using it to estimate the best escape route given the possibility that the fire can spread. Additionally they described how to avoid excessive crowding, this is done by giving each edge a maximum capacity. If an edge is at maximum capacity it is unusable and people would be given different escape routes.

The research paper A Spatio-temporal Probabilistic Model of Hazard and Crowd Dynamics in Disasters for Evacuation Planning \cite{dbn} by O.-C. Granmo et al. proposes a model that "integrates crowd with hazard dynamics". In their scenario they use a fire on board a ship to act as an uncertain and dynamic environment.  ACO is used to find the paths with least risk associated with it and in their simulations ACO "finds these safes routes with probability close to 1.0 using merely 10 ants". While their graph only contains 9 nodes it still shows that with few ants ACO can find the best routes.  

In the research paper An improved ant colony optimization algorithm for solving a complex combinatorial optimization problem \cite{Yang2010653} J. Yang et al. writes that their improved version of an Ant Colony Optimization algorithm has " much higher convergence speed than that of genetic algorithm (GA), simulated annealing (SA), and basic ant colony algorithm, and can jump over the region of the local minimum, and escape from the trap of a local minimum successfully and achieve the best solutions". This is done by a combination of ACO and a genetic algorithm. In the beginning of the simulations basic ACO can end up sending ants via a less than optimal path and the pheromones will reinforce this path. By using a mutation operator ACO can avoid the local minimums as it can introduce paths that are being ignored.

\section{Ship Evacuation}

During an evacuation speed is off the essence. The International Maritime Organization (IMO) is an United Nation's agency responsible for
the safety and security of shipping. They require, under SOLAS Chapter III Regulation 21.1.4 \cite{imo}, that: 
\begin{quotation}
\textit{All survival craft required to provide for abandonment by the total number of persons on board shall be capable of being launched with their full complement of persons and equipment within a period of 30 min from the time the abandon ship signal is given after all persons have been assembled, with lifejackets donned.}
\end{quotation}
Additionally, they recommend that the total evacuation time is a maximum of 60 minutes for ships with up to three vertical fire zones and
80 minutes for ships with more than three vertical fire zones \cite{total}. This leaves either 30 or 50 minutes, depending on ship size, to detect the crisis and get the passengers to the embarkation stations and equip them with lifejackets.

The time it takes for each passenger to reach the embarkation stations is calculated by $T = (\gamma + \delta) t$ where $\gamma$ is a factor that determines how safe the path is, $\delta$ is the counterflow factor and $t$ is the travel time in ideal conditions. Factors that can hinder the flow of passengers includes: high occupant density, closed doors, debris, ship motion, etc. Thus the travel time is the amount of time it would take for each passenger to evacuate given perfect conditions multiplied by all factors that restricts their movements. 

\section{Dijkstra's algorithm}
Dijkstra's Algorithm is a graph search algorithm, created by Edsger Dijkstra in 1959 \cite{Misa}, that finds the shortest path through a graph from a source node to a goal node. The algorithm is commonly used in network routing protocols to find the least costly path through a network. The algorithm can be used to find the shortest path from the source node to a single goal node or from the source node to every other node in the graph. 

The first step in the algorithm is to set the initial distance to every node to infinity, this is done to signify that the node has yet to been visited. During the first iteration of the algorithm the
source node is the current node and the distance between them is 0, seeing as they are in fact the same node. During the following iterations the closest unvisited node is selected as the
current node.

\begin{figure}[h]
\centering
\begin{math}
d(u) = d(v) + d(v,u)
\end{math}
\caption{The shortest distance to the unvisted node from the source node via the visited node}
\label{fig:relax}
\end{figure}

From the current node, follow the edges leading out of the node and calculate the distance to every unvisited node and update the minimum distance in these nodes. Minimum distance meaning the shortest distance from the source node to the unvisited node. This is calculated by adding the sum of the distance between the source node and the current node with the distance to the unvisited node as seen in figure \ref{fig:relax}. If a shorter path via another node is discovered in later iterations the minimum distance is updated. After calculating the distance to all nearby nodes the current node is marked as visited and the node with the lowest minimum distance connected to the visited node is selected as the new current node. 

The process is complete when the goal node(s) is marked as visited. It is worth considering that Dijkstra's algorithm does not attempt to find the goal node, it simply attempts to find the best next step and simply stops when it reaches its goal. The most basic implementation of Dijkstra's algorithm runs at $O(|N|^2)$ where $N$ is the amount of nodes in the graph. In these basic implementations the nodes are stored in an array or linked list. In 1984 Michael Lawrence Fredman and Robert E. Tarjan published the paper Fibonacci Heaps And Their Uses In Improved Network Optimization Algorithms \cite{715934}, in which it was proven that the Dikjstra's algorithm running time could be reduced to $O(|E| + |N|log|N|)$ where $E$ is the number of edges in the graph. The Fibonacci Heap is used as a priority queue, meaning a stack data structure where the elements with the highest priority are retrieved first. However this is only an improvement over the basic implementation if the amount of edges out of each node is significantly lower than the amount of nodes in total.

In the research paper The Application of Dijkstra's Algorithm in the Intelligent Fire Evacuation System \cite{6305611} writen by X. Yuanzhe et al. Dijkstra's algorithm is used in an evacuation system. In the paper they write "A* algorithm aims to solving the shortest distance between one point to another. Dijkstra's algorithm aims to solving the shortest distance between one point to the remaining points. While Floyd's algorithm aims to solving the shortest distance between two of all points in the network. Thus Dijkstra's algorithm is one of the best methods for solving the evacuation safely". Thus, according to the writers, Dijkstra's algorithm is the best suited classic shortest path algorithm and should serve as an appropriate comparison to ACO.








