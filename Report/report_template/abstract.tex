\begin{abstract}

Evacuating passengers during a fire on board a ship is a difficult task and any improvements on current procedures can help save lives. This report describes how an Ant Colony Optimization (ACO) pathfinding algorithm could possibly be used to lead passengers out of this dangerous situation. The ships were modeled as graphs with nodes and vertices. In the simulation passengers were equipped with a smart phone running an application which showed them the way out. The passengers could end up panicking given close proximity to fire or other stressing factors, in which case they would stop following directions. Additionally, high density of passengers in rooms and corridors slowed down the speed of evacuation.  The results produced by ACO were compared to Dijkstra's pathfinding algorithm and were promising. They showed that ACO performed well in dynamic environments and could be used in a crisis situation to guide people out of danger.

\end{abstract}
