\chapter{Introduction}
\label{ch:introduction}

\section{Introduction}

A fire onboard a ship can have disastrous consequences, and as good as the evacuation plans may be now,                           
they will need improvement to ensure that as many passengers as possible survives. This project will work                           
 on modifying existing algorithms that will find the optimal path of evacuation during a simulated crisis, and which could         
 one day be used to help save lives during a crisis that otherwise may have been lost. 

Our main focus is the Ant Colony Optimization, a pathfinding algorithm which finds a path through
 the environment to the desired destination by mimicking the behavior of an ant colony. The algorithm sends    
out virtual ants that search for a valid path then report back the path they took.            
These paths are marked with pheromones and any consecutive ants are more likely to walk on the previously successful paths.

In this thesis we will compare Dijkstra's pathfinding algorithm to ACO, as Dijkstra's is one of the most commonly
 used pathfinding algorithms today. We will model the ship as a dynamic graph with nodes and edges where
 the nodes can be rooms, hallways or other sections of the ships and the edges indicate that it is possible to
 transition from one node to another. 

Additionally, we will simulate human behavior during a crisis and observe
 how this affects the algorithms. Finally we will look into how high occupant density around exits slow down the 
evacuation process and explore how to redirect passengers.

\section{Motivation}

Every year people die in boating accidents, we hope that our work can be                                                                         
used to save future lives. Jim Hall, the Chairman of the National Transportation 
Safety Board testified in front of the Subcommittee on Coast Guard and Maritime 
Transportation, making several safety recomendations after the Scandinavian Star
accident \cite{ntsb}. One of them were to "Improve crew language/communication 
ability to assist passengers during emergencies." Our work, which will be part of a system
that will provide passengers with the quickest route to the exit, could output 
any directions in the language of the phone owner. This would not only improve 
communication if the employees and passengers do not speak the same language, 
it would also ensure that the directives reach all passengers quicker than if the employees 
were to guide all passengers. Additionally it increases the probability that all passengers 
receive proper guidance. For instance any message played over a sound system could be
difficult to hear during a panic and passengers could be in rooms or corridors
where they would be unrachable for the employees.

%\section{Goal}

%\subsection{Field of research}

\section{Problem Statement}

We aim to develop and approach several algorithms and determine which algorithm will provide us with the         
best evacuation route. Our hypothesis is that Ant Colony Optimization (ACO) will perform better than the 
most commonly used pathfinding algorithms. Performance is measured by how many passengers 
survive the crisis situation and how efficient the algorithms are. 

Secondly, we hypothesize that ACO will outperform Dijkstra's pathfinding algorithm       
given a continuously changing graph. In a real crisis situation the environment is subject
to change. For instance a corridor that previously was safe can be filled with smoke
or a room that was empty can become filled with people and be undesirable as a path
to the exit.

Thirdly, we hypothesize that ACO will achieve a better outcome than Dijkstra's pathfinding algorithm
given high occupant density around exits. High occupant density is dangerous not only
because it will slow the evacuation speed it also creates potentially harmful situations. Thus the algorithm
should predict possible future bottlenecks and redirect passengers as needed.

Finally, we hypothesize that ACO will produce better results than Dijkstra's pathfinding algorithm                 
by adding human behavior. To produce accurate results the algorithms needs to be subjected
to a model that is as close to reality as possible and it then becomes necessary to include
human behavior to the model. Passengers can ignore directions, misunderstand directions, panic
and so forth. Behavior is different if the passenger is alone, with his or her family, or in a large group;
and thus it becomes important to account for individuals that may not follow the directions perfectly.

\section{Limitations and key assumptions}

To properly define the scope of our master thesis we have outlined                                                                               %Rewrite this section
some limitations and assumptions. We will limit ourselves to testing and modifying algorithms;                                  
we will base ourselves on well known algorithms, not create a fully fleshed out system that would distribute the 
safest paths to the passenger nor create the application that will run on the passenger's smartphones.
The focus in our master thesis will be on the algorithms themselves and not the final system as that have a large set
of challenges itself. For instance the complete system would need to find ways to distribute directions to each phone,
discover which phones are within the crisis area, receive information about the location of the fire onboard the ship etc.

We will apply current knowledge of human behavior during crisis. The field of human behavior during a crisis is large
and expanding it would be outside of our field of expertise. And as such we will use existing models and knowledge
to simulate the demeanor of the passengers. Additionally we assume that human behavior can be realistically modeled
and that such a framework can capture the complexity of passengers in a crisis.

A high occupant density around exists will quickly slow down evacuation speed and as such there have been conducted several studies
towards investigating the dynamics around evacuations. As these studies have yielded promising results we will
limit ourselves to using these results and not producing our own. We do not find it necessary to focus on this particular segment 
of the evacuation process as we wish to have a broader view of the entire evacuation development.

We assume a ship with hazards can be sufficiently modeled as a graph with nodes and edges. As a ship is a very complex
environment it is possible that by reducing a complex structure into a simpler model some details would be lost in the
translation. For instance a table could be knocked over or a spill could cause someone to slip. However, as there are a large number of 
passengers on large ships we expect that by simply adding some regulatory factors to each area in the ship and running
several iterations the results will be fairly realistic.                                                           %In line with existing ship standards and regulations                                                                    %Is this a good place to mention our nodes inside nodes plan? -> yes it is

\section{Contributions}

 We will make improvements to already existing pathfinding algorithms that can be used to find a path 
for passengers evacuating a ship. Given that our hypothesis is proven
correct our then modified version of ACO will outperform conventional pathfinding algorithms in our 
specific scenario. This in turn could have wider applications as there are multiple scenarios
where finding a path is subject to change. 

We will determine how human behaviour and occupant density around exits affects path finding algorithms.
It is easy to imagine how a panic around an evacuation bottleneck will become dangerous and slow
down the rate of egress even further than it normaly would. We will attempt to find ways to redirect
evacuees before the issue occurs. If there are a large amount of people around a bottleneck at the
start of the evacuation we will redistribute the passengers to quickly solve the problem.

%\section{Target audience}

\section{Report outline}

The master thesis contains 4 chapters: Introduction, Methods, Results and Conclusion.               
The introduction starts by broadly describing the thesis in simple terms, then explains the motivation behind the project,
the goal of the thesis, the current state of the art and finally the problem statement. 

The Methods describes the methods we have used in our research, why we used them and what makes
our research different from the current state of the art. It describes the implementation of the human behavior,
the calculations regarding the evacuation speed and egress restricting factors as well as the implementation of the 
ship model. Furthermore, the methods chapter contains the hypothesis.

The Results presents the results and the evidence that either supports or oppose our hypothesis. The
statistical data gathered using the methods in the previous chapter is displayed and any issues we have experienced
is explained.

The Discussion looks at what the data means, whether it matters or not and how it compares to other research in the field.
Additionally it discusses how successful the project has been and if ACO is better than conventional algorithms, as we hypothesized.
Finally it contains proposed future steps that can be made to continue developing the project.
