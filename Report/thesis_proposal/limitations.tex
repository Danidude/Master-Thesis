\chapter{Limitations and key assumptions}
\label{ch:limitations}

%1-2 pages
%Fits into the method chapter in the thesis.
%Make it clear what you will not do.
%Make it clear what you need to assume to carry out your
%master thesis.

%Limitation: Example
%Applying NN algorithms to governmental web sites.
%Not personal web sites.
%Not business web sites.
%Modelling the development cycle in mobile software developments.
%Not application development.
%Not deployment.

%Assumptions: Example
%Assume that a web site can be modelled as a graph with
%nodes and vertices.
%Assume that wireless data can be modelled using statistics.
%Assume that useful data can be collected from the network
%using sniffer applications.

%Wrong assumptions
%Assume that I will answer my problem deffinitions. 
%Assume that I will prove my hypthesis.

To properly define the scope of our master thesis we have outlined
some limitations and assumptions. We will limit ourselves to testing and modifying algorithms;
we will not create completely new algorithms, a fully fleshed out system that would distribute the
safest paths to the passenger nor will we be creating the application that would run on the passenger's smartphones.
The focus in our master thesis will be on the algorithms themselves and not the final system as that have a large set
of challenges itself. For instance the complete system would need to find ways to distribute directions to each phone,
discover which phones are within the crisis area, receive information about the location of the fire onboard the ship etc.

We will apply current knowledge of human behavior during crisis. The field of human behavior during a crisis is large
and expanding it would be outside of our field of expertise. And as such we will use existing models and knowledge
to simulate the demeanor of the passengers. Additionally we assume that human behavior can be realistically modeled
and that such a framework can capture the complexity of passengers in a crisis.

A high occupant density around exists will quickly slow down evacuation speed and as such there have been conducted several studies
towards investigating the dynamics around evacuations. As these studies have yielded promising results we will
limit ourselves to using these results and not producing our own. We do not find it necessary to focus on this particular segment 
of the evacuation process as we wish to have a broader view of the entire evacuation development.

We assume a ship with hazards can be sufficiently modeled as a graph with nodes and vertices. As a ship is a very complex
environment it is possible that by reducing a complex structure into a simpler model some details would be lost in the
translation. For instance a table could be knocked over or a spill could cause someone to slip. However, as there are a large number of 
passengers on large ships we expect that by simply adding some regulatory factors to each area in the ship and running
several iterations the results will be fairly realistic.
